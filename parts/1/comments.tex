\chapterimage{chapter_head_3.png}

\chapter{Comments}

\section{Comments Do Not Make Up for Bad Code}\index{Comments!Comments Do Not Make Up for Bad Code}

One of the more common motivations for writing comments is bad code. We write a module
and we know it is confusing and disorganized. We know it's a mess. So we say to ourselves,
“Ooh, I'd better comment that!” No! You'd better clean it!
Clear and expressive code with few comments is far superior to cluttered and complex
code with lots of comments. Rather than spend your time writing the comments that
explain the mess you've made, spend it cleaning that mess.

\section{Explain Yourself as Code}\index{Comments!Explain Yourself in Code}

There are certainly times when code makes a poor vehicle for explanation. Unfortunately, many programmers have taken this to mean that code is seldom, if ever, a good means for explanation. This is patently false. Which would you rather see? This:

\begin{tcolorbox}[breakable, colback=blue!10!white, colframe=blue!85!black]

\begin{lstlisting}[language = java, basicstyle=\small]
// Check to see if the employee is eligible for full benefits
if ((employee.flags & HOURLY_FLAG) && (employee.age > 65))
\end{lstlisting}

\tcblower

\begin{lstlisting}[language = java, basicstyle=\small]
if (employee.isEligibleForFullBenefits())
\end{lstlisting}

\end{tcolorbox}

It takes only a few seconds of thought to explain most of your intent in code. In many cases it's simply a matter of creating a function that says the same thing as the comment you want to write.

\begin{marker}
If there is a comment for a piece of logic, it might be a sign of review comments saying "this should be turned into a method"
\end{marker}

\section{Good Comments}\index{Comments!Good Comments}

Some comments are necessary or beneficial. The only truly good comment is the comment you found a way not to write.

\subsection{Legal Comments}

\begin{tcolorbox}[breakable, colback=green!10!white, colframe=green!85!black, title = An Example]

\begin{verbatim}
/*
 * Copyright <some author>
 *
 * Licensed under the Apache License, Version 2.0 (the "License");
 * you may not use this file except in compliance with the License.
 * You may obtain a copy of the License at
 *
 *     http://www.apache.org/licenses/LICENSE-2.0
 *
 * Unless required by applicable law or agreed to in writing, software
 * distributed under the License is distributed on an "AS IS" BASIS,
 * WITHOUT WARRANTIES OR CONDITIONS OF ANY KIND, either express or
 * implied. See the License for the specific language governing
 * permissions and limitations under the License.
 */
\end{verbatim}

\end{tcolorbox}

\subsection{Informative Comments}

\subsubsection{Regular Expression}

Regex is hard to read. A comment tells what that regex does in an easy way. For example, 

\begin{tcolorbox}[breakable, colback=green!10!white, colframe=green!85!black, sidebyside, righthand width = 3cm, tikz lower, label=blocks-and-indenting-good]

\begin{lstlisting}[language = java, basicstyle=\small]
// format matched kk:mm:ss EEE, MMM dd, yyyy
Pattern timeMatcher = Pattern.compile(
        "\\d*:\\d*:\\d* \\w*, \\w* \\d*, \\d*"
);
\end{lstlisting}

\tcblower

\path[fill = yellow, draw = yellow!75!red] (0, 0) circle (1cm);
\fill[red] (45:5mm) circle (1mm);
\fill[red] (135:5mm) circle (1mm);
\draw[line width=1mm,red] (215:5mm) arc (215:325:5mm);

\end{tcolorbox}

In this case the comment lets us know that the regular expression is intended to match a time and date that were formatted with the SimpleDateFormat.format function using the specified format string. Still, it might have been better, and clearer, if this code had been moved to a special class that converted the formats of dates and times. Then the comment would likely have been superfluous.

\subsection{Explanation of Intent}

Sometimes a comment goes beyond just useful information about the implementation and
provides the intent behind a decision. For example:

\begin{tcolorbox}[breakable, colback=green!10!white, colframe=green!85!black, sidebyside, righthand width = 3cm, tikz lower, label=blocks-and-indenting-good]

\begin{lstlisting}[language = java, basicstyle=\small]
// This is our best attempt to get a race condition
// by creating large number of threads.
for (int i = 0; i < 25000; i++) {
    WidgetBuilderThread widgetBuilderThread = new WidgetBuilderThread(widgetBuilder, text, parent, failFlag);
    Thread thread = new Thread(widgetBuilderThread);
    thread.start();
}
\end{lstlisting}

\tcblower

\path[fill = yellow, draw = yellow!75!red] (0, 0) circle (1cm);
\fill[red] (45:5mm) circle (1mm);
\fill[red] (135:5mm) circle (1mm);
\draw[line width=1mm,red] (215:5mm) arc (215:325:5mm);

\end{tcolorbox}

\subsection{Clarification}

Sometimes it is just helpful to translate the meaning of some obscure argument or return value into something that's readable. In general it is better to find a way to make that argument or return value clear in its own right; but when its part of the standard library, or in code that you cannot alter, then a helpful clarifying comment can be useful.

\begin{tcolorbox}[breakable, colback=green!10!white, colframe=green!85!black, sidebyside, righthand width = 3cm, tikz lower, label=blocks-and-indenting-good]

\begin{lstlisting}[language = java, basicstyle=\small]
public void testCompareTo() throws Exception
{
    WikiPagePath a = PathParser.parse("PageA");
    WikiPagePath ab = PathParser.parse("PageA.PageB");
    WikiPagePath b = PathParser.parse("PageB");
    WikiPagePath aa = PathParser.parse("PageA.PageA");
    WikiPagePath bb = PathParser.parse("PageB.PageB");
    WikiPagePath ba = PathParser.parse("PageB.PageA");
    
    assertTrue(a.compareTo(a) == 0); // a == a
    assertTrue(a.compareTo(b) != 0); // a != b
    assertTrue(ab.compareTo(ab) == 0); // ab == ab
    assertTrue(a.compareTo(b) == -1); // a < b
    assertTrue(aa.compareTo(ab) == -1); // aa < ab
    assertTrue(ba.compareTo(bb) == -1); // ba < bb
    assertTrue(b.compareTo(a) == 1); // b > a
    assertTrue(ab.compareTo(aa) == 1); // ab > aa
    assertTrue(bb.compareTo(ba) == 1); // bb > ba
}
\end{lstlisting}

\tcblower

\path[fill = yellow, draw = yellow!75!red] (0, 0) circle (1cm);
\fill[red] (45:5mm) circle (1mm);
\fill[red] (135:5mm) circle (1mm);
\draw[line width=1mm,red] (215:5mm) arc (215:325:5mm);

\end{tcolorbox}

\subsection{Warning of Consequences}

Sometimes it is useful to warn other programmers about certain consequences. For example, you might want to put "Do not use Google API for this method because it cannot handle such such business case"

\subsection{TODO Comments}

TODO's are jobs that the programmer thinks should be done, but for some reason
can't do at the moment.

Nowadays, most good IDEs provide special gestures and features to locate all the
TODO comments, so it's not likely that they will get lost. Still, you don't want your code to be littered with TODOs. So scan through them regularly and eliminate the ones you can.

\begin{marker}
When you see a relevant TODO during review process, see if this can be resoved in the same PR
\end{marker}

\subsection{Javadocs in Public APIs}

There is nothing quite so helpful and satisfying as a well-described public API. If you are writing a public API, then you should certainly write good javadocs for it. But keep in mind the rest of the advice in this chapter. Javadocs can be just as misleading, nonlocal, and dishonest as any other kind of comment.

\begin{marker}
Druing review process, check if implementation matches what Javadocs describes .
\end{marker}

\section{Bad Comments}\index{Comments!Bad Comments}

\subsection{Confusing Comments}

When you review someone else's comments and pauses, this might be a sign of incomplete or insufficient comments. You should as the code author to elaborate it more on that.

\subsection{Redundant Comments}

\begin{tcolorbox}[breakable, colback=red!10!white, colframe=red!85!black, sidebyside, righthand width = 3cm, tikz lower, label=redundant-comment-bad]

\begin{lstlisting}[language = java, basicstyle=\small]
// Utility method that returns when this.closed is true. Throws an exception
// if the timeout is reached.
public synchronized void waitForClose(final long timeoutMillis) throws Exception {
    if(!closed)
    {
        wait(timeoutMillis);
        if(!closed) {
            throw new Exception("MockResponseSender could not be closed");
        }
    }
}
\end{lstlisting}

\tcblower

\path[fill = yellow, draw = yellow!75!red] (0, 0) circle (1cm);
\fill[red] (45:5mm) circle (1mm);
\fill[red] (135:5mm) circle (1mm);
\draw[line width=1mm,red] (230:6mm) arc (145:35:5mm);

\end{tcolorbox}

Listing~\ref{redundant-comment-bad} shows a simple function with a header comment that is completely redundant. The comment probably takes longer to read than the code itself.

What purpose does this comment serve? It's certainly not more informative than the code. It does not justify the code, or provide intent or rationale. It is not easier to read than the code.

\begin{marker}
Method comments
\begin{itemize}
    \item should be more informative than the code
    \item Justify the code, i.e. provide intent or rationale
    \item must be easier to read than the code
\end{itemize}
\end{marker}

Now consider the legion of useless and redundant javadocs in Listing 4-2 taken from
Tomcat. These comments serve only to clutter and obscure the code. They serve no documentary purpose at all

\begin{tcolorbox}[breakable, colback=red!10!white, colframe=red!85!black, sidebyside, righthand width = 3cm, tikz lower, label=redundant-comment-bad-repeating]

\begin{lstlisting}[language = java, basicstyle=\small]
public abstract class ContainerBase
implements Container, Lifecycle, Pipeline,
MBeanRegistration, Serializable {
    /**
    * The processor delay for this component.
    */
    protected int backgroundProcessorDelay = -1;
    
    /**
    * The lifecycle event support for this component.
    */
    protected LifecycleSupport lifecycle = new LifecycleSupport(this);
    
    /**
    * The container event listeners for this Container.
    */
    protected ArrayList listeners = new ArrayList();
    
    /**
    * The Loader implementation with which this Container is
    * associated.
    */
    protected Loader loader = null;
    
    /**
    * The Logger implementation with which this Container is
    * associated.
    */
    protected Log logger = null;
    
    /**
    * Associated logger name.
    */
    protected String logName = null;
\end{lstlisting}

\tcblower

\path[fill = yellow, draw = yellow!75!red] (0, 0) circle (1cm);
\fill[red] (45:5mm) circle (1mm);
\fill[red] (135:5mm) circle (1mm);
\draw[line width=1mm,red] (230:6mm) arc (145:35:5mm);

\end{tcolorbox}

\begin{marker}
Avoid the comments that says the name of a variable in anothe way like the ones shown in Listing~\ref{redundant-comment-bad-repeating}
\end{marker}

In addition, those are "noisy" comments that you should really comment on during code review:

\begin{tcolorbox}[breakable, colback=red!10!white, colframe=red!85!black, sidebyside, righthand width = 3cm, tikz lower]

\begin{lstlisting}[language = java, basicstyle=\small]
/**
* Default constructor.
*/
protected AnnualDateRule() {
}

/**
 * The day of the month.
 */
private int dayOfMonth;

/**
 * Returns the day of the month.
 *
 * @return the day of the month.
 */
public int getDayOfMonth() {
    return dayOfMonth;
}

\end{lstlisting}

\tcblower

\path[fill = yellow, draw = yellow!75!red] (0, 0) circle (1cm);
\fill[red] (45:5mm) circle (1mm);
\fill[red] (135:5mm) circle (1mm);
\draw[line width=1mm,red] (230:6mm) arc (145:35:5mm);

\end{tcolorbox}

\subsection{Don't Use a Comment When You Can Use a Function or a Variable}

During review process, if you see the author put a comment on top of a section of code, this is a sign of potentially saying something on that: see if the comment can be replaced by a set of more expressive code. For example, consider the following stretch of code:

\begin{tcolorbox}[breakable, colback=red!10!white, colframe=red!85!black, sidebyside, righthand width = 3cm, tikz lower]

\begin{lstlisting}[language = java, basicstyle=\small]
// does the module from the global list <mod> depend on the
// subsystem we are part of?
if (smodule.getDependSubsystems().contains(subSysMod.getSubSystem()))
\end{lstlisting}

\tcblower

\path[fill = yellow, draw = yellow!75!red] (0, 0) circle (1cm);
\fill[red] (45:5mm) circle (1mm);
\fill[red] (135:5mm) circle (1mm);
\draw[line width=1mm,red] (230:6mm) arc (145:35:5mm);

\end{tcolorbox}

This could be rephrased without the comment as

\begin{tcolorbox}[breakable, colback=green!10!white, colframe=green!85!black, sidebyside, righthand width = 3cm, tikz lower]

\begin{lstlisting}[language = java, basicstyle=\small]
ArrayList moduleDependees = smodule.getDependSubsystems();
String ourSubSystem = subSysMod.getSubSystem();
if (moduleDependees.contains(ourSubSystem))
\end{lstlisting}

\tcblower

\path[fill = yellow, draw = yellow!75!red] (0, 0) circle (1cm);
\fill[red] (45:5mm) circle (1mm);
\fill[red] (135:5mm) circle (1mm);
\draw[line width=1mm,red] (215:5mm) arc (215:325:5mm);

\end{tcolorbox}

\begin{marker}
On seeing a comment, see if that comment can be replaced by a set more expressive code
\end{marker}

\subsection{Position Markers}

\begin{tcolorbox}[breakable, colback=red!10!white, colframe=red!85!black, sidebyside, righthand width = 3cm, tikz lower, title = Example of position markers]

\begin{lstlisting}[language = java, basicstyle=\small]
// Actions //////////////////////////////////
\end{lstlisting}

\tcblower

\path[fill = yellow, draw = yellow!75!red] (0, 0) circle (1cm);
\fill[red] (45:5mm) circle (1mm);
\fill[red] (135:5mm) circle (1mm);
\draw[line width=1mm,red] (230:6mm) arc (145:35:5mm);

\end{tcolorbox}

\begin{marker}
There are rare times when it makes sense to gather certain functions together beneath a banner like this. But in general they are clutter that should be eliminated
\end{marker}

\subsection{Closing Brace Comments}

\begin{tcolorbox}[breakable, colback=red!10!white, colframe=red!85!black, sidebyside, righthand width = 3cm, tikz lower, title = Example of closing brace comments]

\begin{lstlisting}[language = java, basicstyle=\small]
public static void main(String[] args) {
    BufferedReader in = new BufferedReader(new InputStreamReader(System.in));
    String line;
    int lineCount = 0;
    int charCount = 0;
    int wordCount = 0;
    try {
        while ((line = in.readLine()) != null) {
            lineCount++;
            charCount += line.length();
            String words[] = line.split("\\W");
            wordCount += words.length;
        } //while
    
        System.out.println("wordCount = " + wordCount);
        System.out.println("lineCount = " + lineCount);
        System.out.println("charCount = " + charCount);
    } // try
    catch (IOException e) {
        System.err.println("Error:" + e.getMessage());
    } //catch
} //main
\end{lstlisting}

\tcblower

\path[fill = yellow, draw = yellow!75!red] (0, 0) circle (1cm);
\fill[red] (45:5mm) circle (1mm);
\fill[red] (135:5mm) circle (1mm);
\draw[line width=1mm,red] (230:6mm) arc (145:35:5mm);

\end{tcolorbox}

Although this might make sense for long functions with deeply nested structures, it serves only to clutter the kind of small and encapsulated functions that we prefer. So if you find yourself wanting to mark your closing braces, try to shorten your functions instead.

\begin{marker}
When you see a closing brace comment, this is a sign of suggesting shortening functions
\end{marker}

\subsection{Attributions and Bylines}

\begin{tcolorbox}[breakable, colback=red!10!white, colframe=red!85!black, sidebyside, righthand width = 3cm, tikz lower, title = Example of closing brace comments]

\begin{lstlisting}[language = java, basicstyle=\small]
/* Added by Rick */
\end{lstlisting}

\tcblower

\path[fill = yellow, draw = yellow!75!red] (0, 0) circle (1cm);
\fill[red] (45:5mm) circle (1mm);
\fill[red] (135:5mm) circle (1mm);
\draw[line width=1mm,red] (230:6mm) arc (145:35:5mm);

\end{tcolorbox}

Source code control systems are very good at remembering who added what, when.
There is no need to pollute the code with little bylines.

\subsection{Commented-Out Code}

\begin{tcolorbox}[breakable, colback=red!10!white, colframe=red!85!black, sidebyside, righthand width = 3cm, tikz lower]

\begin{lstlisting}[language = java, basicstyle=\small]
InputStreamResponse response = new InputStreamResponse();
response.setBody(formatter.getResultStream(), formatter.getByteCount());
// InputStream resultsStream = formatter.getResultStream();
// StreamReader reader = new StreamReader(resultsStream);
// response.setContent(reader.read(formatter.getByteCount()));
\end{lstlisting}

\tcblower

\path[fill = yellow, draw = yellow!75!red] (0, 0) circle (1cm);
\fill[red] (45:5mm) circle (1mm);
\fill[red] (135:5mm) circle (1mm);
\draw[line width=1mm,red] (230:6mm) arc (145:35:5mm);

\end{tcolorbox}

Few practices are as odious as commenting-out code. Don‘t do this!

There was a time, back in the sixties, when commenting-out code might have been
useful. But we've had good source code control systems for a very long time now. Those systems will remember the code for us. We don't have to comment it out any more. Just delete the code. We won't lose it. Promise.

 \begin{marker}
 Ask PR author to remove any commented out code, if any
 \end{marker}
 
 \subsection{HTML Comments}
 
 HTML in source code comments is an abomination, as you can tell by reading the code below. It makes the comments hard to read in the one place where they should be easy to read—the editor/IDE. If comments are going to be extracted by some tool (like Javadoc) to appear in a Web page, then it should be the responsibility of that tool, and not the programmer, to adorn the comments with appropriate HTML.
 
 \begin{tcolorbox}[breakable, colback=red!10!white, colframe=red!85!black, sidebyside, righthand width = 3cm, tikz lower]

\begin{lstlisting}[language = java, basicstyle=\small]
/**
 * Task to run fit tests.
 * This task runs fitnesse tests and publishes the results.
 * <p/>
 * <pre>
 * Usage:
 * &lt;taskdef name=&quot;execute-fitnesse-tests&quot;
 * classname=&quot;fitnesse.ant.ExecuteFitnesseTestsTask&quot;
 * classpathref=&quot;classpath&quot; /&gt;
 * OR
 * &lt;taskdef classpathref=&quot;classpath&quot;
 * resource=&quot;tasks.properties&quot; /&gt;
 * <p/>
 * &lt;execute-fitnesse-tests
 * suitepage=&quot;FitNesse.SuiteAcceptanceTests&quot;
 * fitnesseport=&quot;8082&quot;
 * resultsdir=&quot;${results.dir}&quot;
 * resultshtmlpage=&quot;fit-results.html&quot;
 * classpathref=&quot;classpath&quot; /&gt;
 * </pre>
 */
\end{lstlisting}

\tcblower

\path[fill = yellow, draw = yellow!75!red] (0, 0) circle (1cm);
\fill[red] (45:5mm) circle (1mm);
\fill[red] (135:5mm) circle (1mm);
\draw[line width=1mm,red] (230:6mm) arc (145:35:5mm);

\end{tcolorbox}

\subsection{Nonlocal Information}

If you must write a comment, then make sure it describes the code it appears near. Don't offer systemwide information in the context of a local comment. Consider, for example, the javadoc comment below. Aside from the fact that it is horribly redundant, it also offers information about the default port. And yet the function has absolutely no control over what that default is. The comment is not describing the function, but some other, far distant part of the system. Of course there is no guarantee that this comment will be changed when the code containing the default is changed.

\begin{tcolorbox}[breakable, colback=red!10!white, colframe=red!85!black, sidebyside, righthand width = 3cm, tikz lower]

\begin{lstlisting}[language = java, basicstyle=\small]
/**
* Port on which fitnesse would run. Defaults to <b>8082</b>.
*
* @param fitnessePort
*/
public void setFitnessePort(int fitnessePort) {
    this.fitnessePort = fitnessePort;
}
\end{lstlisting}

\tcblower

\path[fill = yellow, draw = yellow!75!red] (0, 0) circle (1cm);
\fill[red] (45:5mm) circle (1mm);
\fill[red] (135:5mm) circle (1mm);
\draw[line width=1mm,red] (230:6mm) arc (145:35:5mm);

\end{tcolorbox}

\subsection{Inobvious Connection}

Lets consider this example:

\begin{tcolorbox}[breakable, colback=red!10!white, colframe=red!85!black, sidebyside, righthand width = 3cm, tikz lower]

\begin{lstlisting}[language = java, basicstyle=\small]
/*
* start with an array that is big enough to hold all the pixels
* (plus filter bytes), and an extra 200 bytes for header info
*/
this.pngBytes = new byte[((this.width + 1) * this.height * 3) + 200];
\end{lstlisting}

\tcblower

\path[fill = yellow, draw = yellow!75!red] (0, 0) circle (1cm);
\fill[red] (45:5mm) circle (1mm);
\fill[red] (135:5mm) circle (1mm);
\draw[line width=1mm,red] (230:6mm) arc (145:35:5mm);

\end{tcolorbox}

What is filter byte? The code itself has nothing to do with filter byte. 

The connection between a comment and the code it describes should be obvious. If you are going to the trouble to write a comment, then at least you'd like the reader to be able to look at the comment and the code and understand what the comment is talking about.