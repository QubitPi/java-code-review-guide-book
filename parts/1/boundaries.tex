\chapterimage{chapter_head_4.png}

\chapter{Boundaries}

Interesting things happen at boundaries(e.g. between our code and 3rd-party libraries). Change(e.g. from a 3rd-party library) is one of those things. Good software designs accommodate change without huge investments and rework. When we use code that is out of our control, special care must be taken to protect our investment and make sure future change is not too costly.

Code at the boundaries needs clear separation and tests that define expectations. We
should avoid letting too much of our code know about the third-party particulars. It's better to depend on something you control than on something you don't control, lest it end up controlling you.

We manage third-party boundaries by having very few places in the code that refer to
them. We may wrap them in our own object, or we may use an ADAPTER to convert from
our perfect interface to the provided interface. Either way our code speaks to us better, promotes internally consistent usage across the boundary, and has fewer maintenance points when the third-party code changes. The sections below discusses how to achieve this in detail and, of course, guides your review process.

\section{Using Third-Party Code}\index{Boundaries!Using Third-Party Code}

There is a natural tension between the provider of an interface and the user of an interface. Providers of third-party packages and frameworks strive for broad applicability so they can work in many environments and appeal to a wide audience. Users, on the other hand, want an interface that is focused on their particular needs. This tension can cause problems at the boundaries of our systems.

For example, \inlinecode[green]{java.util.Map} is a very useful object, but what if we would like a map that just supports inserts and retrievals, but not delete? When you pass this map around system, you cannot guarantee that other maintainers would not misused it by calling a delete method.

A cleaner way to use Map, for example, might look like the following:

\begin{tcolorbox}[breakable, colback=green!10!white, colframe=green!85!black, sidebyside, righthand width = 3cm, tikz lower, label=blocks-and-indenting-good]

\begin{lstlisting}[language = java, basicstyle=\small]
public class Sensors {

    private Map sensors = new HashMap();
    public Sensor getById(String id) {
    return (Sensor) sensors.get(id);
}
//snip
}
\end{lstlisting}

\tcblower

\path[fill = yellow, draw = yellow!75!red] (0, 0) circle (1cm);
\fill[red] (45:5mm) circle (1mm);
\fill[red] (135:5mm) circle (1mm);
\draw[line width=1mm,red] (215:5mm) arc (215:325:5mm);

\end{tcolorbox}

\textbf{The interface at the boundary (Map) is hidden}. It is able to evolve with very little impact on the rest of the application. The misuse of calling delete is not possible because you never expose that method in this class. 

This interface is also tailored and constrained to meet the needs of the application. It results in code that is easier to understand and harder to misuse. Therefore this class can enforce design and business rules.

\begin{marker}
Try not to pass Maps (or any other interface at a boundary) around your system. If you use a boundary interface like Map, keep it inside the class, or close family
of classes, where it is used. Avoid returning it from, or accepting it as an argument to, public APIs.
\end{marker}

\section{Exploring and Learning Boundaries}\index{Boundaries!Exploring and Learning Boundaries}

It's not our job to test the third-party code, but it may be in our best interest to write tests for the third-party code we use. Here is why: when we code 3rd-party libraries into our production system, We would not be surprised to find ourselves bogged down in long debugging sessions trying to figure out whether the bugs we are experiencing are in our code or theirs.

Instead of experimenting and trying out the new stuff in our production code, we could write some tests to explore our understanding of the third-party code. Such tests are called \textit{learning tests}(or boundary tests).

\begin{marker}
Write learning tests to make sure 3rd-parity API does the right things for us. 
\end{marker}

In learning tests we call the third-party API, as we expect to use it in our application. We're essentially doing controlled experiments that check our understanding of that API. The tests focus on what we want out of the API.

To summarize the benefits of learning tests:

\begin{itemize}
    \item Writing learning tests is a very good way to learn a 3rd-party API.
    \item When there are new releases of the third-party package, we run the learning tests to see whether there are behavioral differences.
    \item Learning tests verify that the third-party packages we are using work the way we expect them to.
\end{itemize}


