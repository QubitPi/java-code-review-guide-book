\chapterimage{chapter_head_2.png}

\chapter{Objects and Data Structures}

\section{Data Abstraction}\index{Objects and Data Structures!Data Abstraction}

Data Abstraction represents more than just a data structure. The methods enforce an access policy. For example:

\begin{tcolorbox}[breakable, colback=red!10!white, colframe=red!85!black]
\begin{lstlisting}[language = java, basicstyle=\small]
public interface Point {
    double getX();
    double getY();
    void setCartesian(double x, double y);
    double getR();
    double getTheta();
    void setPolar(double r, double theta);
}
\end{lstlisting}
\end{tcolorbox}

You can read the individual coordinates independently, but \textbf{you must set the coordinates together as an atomic operation}.

\section{\textcolor{red}{Data/Object Anti-Symmetry(VERY IMPORTANT)}}\index{Objects and Data Structures!Data/Object Anti-Symmetry}

We know Java is an \textit{Object Oriented Language} as opposed to \textit{Procedure Language}, such as C. Let's start with 3 data structures and 1 procedure used in procedure language. Listing~\ref{object-and-data-structure-procedure-pair} show the difference between objects and data structures.

\begin{tcolorbox}[breakable, colback=green!10!white, colframe=green!85!black, sidebyside, label = object-and-data-structure-procedure-pair]

\begin{lstlisting}[language = java, basicstyle=\small]
public class Square {
    public Point topLeft;
    public double side;
}

public class Rectangle {
    public Point topLeft;
    public double height;
    public double width;
}

public class Circle {
    public Point center;
    public double radius;
}
\end{lstlisting}

\tcblower

\begin{lstlisting}[language = java, basicstyle=\small]
public class Geometry {
    public final double PI = 3.141592653589793;

    public double area(Object shape) throws NoSuchShapeException {
        if (shape instanceof Square) {
            Square s = (Square)shape;
            return s.side * s.side;
        } else if (shape instanceof Rectangle) {
            Rectangle r = (Rectangle)shape;
            return r.height * r.width;
        } else if (shape instanceof Circle) {
            Circle c = (Circle)shape;
            return PI * c.radius * c.radius;
        }
        throw new NoSuchShapeException();
    }
}
\end{lstlisting}
\end{tcolorbox}

Objects hide their data behind abstractions and expose functions that operate on that data. Data structure expose their data and have no meaningful functions.

The Geometry class operates on the three shape classes. The shape classes are simple data structures without any behavior. All the behavior is in the Geometry class.

Consider what would happen if a \inlinecode[green]{perimeter()} function were added to \inlinecode[green]{Geometry}, you notice the following:

\begin{itemize}
    \item The shape classes would be unaffected. Any other classes that depended upon the shapes would also be unaffected.
    \item On the other hand, if I add a new shape, I must change all the functions in \inlinecode[green]{Geometry} to deal with it.
\end{itemize}

Now hold the points above in your memory for just a moment and let's consider the object-oriented solution in Listing~\ref{object-and-data-structure-oo-solution} 

\begin{tcolorbox}[breakable, colback=green!10!white, colframe=green!85!black, label = object-and-data-structure-oo-solution]
\begin{lstlisting}[language = java, basicstyle=\small]
public class Square implements Shape {
    
    private Point topLeft;
    private double side;

    public double area() {
        return side*side;
    }
}

public class Rectangle implements Shape {
    
    private Point topLeft;
    private double height;
    private double width;

    public double area() {
        return height * width;
    }
}

public class Circle implements Shape {

    public static final double PI = 3.141592653589793;
    
    private Point center;
    private double radius;

    public double area() {
        return PI * radius * radius;
    }
}
\end{lstlisting}
\end{tcolorbox}

When \inlinecode[green]{perimeter()} function is added to \inlinecode[green]{Geometry}, you notice the following:

\begin{itemize}
    \item The shape classes would be affected
    \item On the other hand, if I add a new shape,  \inlinecode[green]{Geometry} is not changed
\end{itemize}

\textit{\textbf{Procedural code (code using data structures) makes it easy to add new functions without changing the existing data structures. OO code, on the other hand, makes it easy to add new classes without changing existing functions.}}

So, the things that are hard for OO are easy for procedures, and the things that are hard for procedures are easy for OO.

In any complex system there are going to be times when we want to add new data
types rather than new functions. For these cases objects and OO are most appropriate. On the other hand, there will also be times when we'll want to add new functions as opposed to data types. In that case procedural code and data structures will be more appropriate.

Mature programmers know that the idea that everything is an object is a myth. Sometimes you really do want simple data structures with procedures operating on them.

\begin{marker}
When we want to add new data types rather than new functions, Objects and OO are better

However if we want to add new functions, procesure code and data structure will be more appropriate

Mature programmers know that the idea that everything is an object is a myth. Sometimes you really do want simple data structures with procedures operating on them
\end{marker}

\section{The Law of Demeter}\index{Objects and Data Structures!The Law of Demeter}

A module should not know about the innards of the objects it manipulates. An object should not expose its internal structure through accessors because to do so is to expose, rather than to hide, its internal structure.

\begin{tcolorbox}[breakable, colback=green!10!white, colframe=green!85!black, center title, title = Law of Demeter]
\begin{center}
A method f of a class C should only call the methods of these:

\begin{itemize}
    \item C
    \item An object created by f
    \item An object passed as an argument to f
    \item An object held in an instance variable of C
    \item The method should not invoke methods on objects that are returned by any of the allowed functions. e.g. final String outputDir = ctxt.getOptions().getScratchDir().getAbsolutePath(); violates this rule
\end{itemize}
\end{center}
\end{tcolorbox}

\section{Data Transfer Objects}\index{Objects and Data Structures!Data Transfer Objects}

The quintessential form of a data structure is a class with public variables and no functions. This is sometimes called a data transfer object, or DTO. \textbf{Somewhat more common in Java is the “bean” form}.
